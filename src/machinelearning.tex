\chapter{Model Selection}
\label{machinelearning}
In Chapter \ref{design} two data represenations were described which could be used in this scenario. To find out which data representation to use the first step is to choose some machine learning models to test on. Selecting the best model is a frequent problem which is still an active area of research. The first step is to use the labelled data to train and test various models. Given the tests results, a model can be selected. However, testing on the same data as was used in training can give results which will not be seen the model is used on new data. Hence it is necessary to split the data into data to be used for training and data to be used for testing. This process is known as cross-validation: the models are being validated against the test set. Cross-validation is most often performed more than once on the same dataset and the results are averaged. Doing so reduces the variance in the performance of the models.
Now that cross-validation has been performed each model has an error estimate. The final model is selected to have the lowest cross-validation error. However, due to the nature of cross-validation this error estimate may not be entirely accurate: we have deliberately chosen the model with the best cross-validation error, so it is possible that this error estimate is optimistic. To get the final error estimate the model is trained on all of the data used during cross-validation and tested on some separate, previously unseen data.
Although it cannot be certain that this model will perform best on new data, it is at least hoped that it will perform similarly to the final error estimate.

\section{Feature selection}
The first action to be taken when creating a model is to determine the features. The features determine the inputs to the model; the data which the model can use to learn and predict. 

\subsection{Question Identity}
The two data representations differ on essentially one issue: is the question identity retained by the input features? One model says yes: retain the identities of the questions by assigning an index of the input vector to the same question each time (which means a new model must be created for each question). The second approach says no: create a moving window of input features from the questions immediately before the question we want to predict. This approach is quite wasteful: If the model is predicting the score for question 20 then it takes the results from questions 14 - 19 and ignores the other 13 questions. The model cannot take all the scores individually or it would be identical to the previous model, so a crude method of using all that extra data is to add an extra feature which is the numerical average of the unused data.

\section{Preliminary models}
Deciding on the models to use is a difficult process. However, since ski-kit learn provides algorithms optimised for speed, most models are very quick to train and test. This allows the option of starting with a variety of models, evaluating how they behave against each other. The models can be split into two categories, depending on the type of data the expect and output.

\subsection{Classifiers}
Classifiers attempt to identify the class a student should be assigned to depending on the input features. Notably, these classes should be discrete but what the models are trying to predict is a continuous percentage. To resolve this problem some sort of discretisation needs to be performed to change a percentage into a class. The granularity of this discretisation is important as it determines the amount of accuracy that can be acheived. 

A natural discretisation is already present in the current design: the visualisation on the webpage. The visualisation uses 9 blocks to display the score. For the sake of this application, further accuracy would be lost when the information is displayed. So, in order to use classifiers on this problem, the output feature (the question we want to predict) will be discretised into one of 9 classes, distributed evenly over the range 0-100%.

\subsection{Regression}
Unlike the classifiers, no adaptations need to be made to the data before supplying it to the regression models.

\subsection{Model definitions}

\paragraph{Logistic Regression}
Although the name is misleading, this model is a classifier. It is originially a binary classifier but this problem has been formulated to use 9 classes. The multivariate implementation used by sci-kit uses the one-vs-all method. It can also use either L1 or L2 regularisation.

\subsubsection{Linear Models}
The following models are all linear regressions: they expect the target value to be a linear combination of the input features. The differences between them occur in the constraints applied to the coefficients.

\paragraph{Linear Regression}
This is the simplest linear model. It uses Ordinary Least Squares to estimate the coefficients of the input vector. This means it tries to minimise the residual sum of squares between the result given by the labelled data and the result predicted by the model.

\paragraph{Ridge Regression}
This model adapts Linear Regression by adding to the minimisation value a penalty depending on the size of the coefficients. Sci-kit allows control of this penalty through the $\alpha$ parameter.

\paragraph{LASSO}
This model applies a similar penalty as Ridge Regression except use the L1 norm instead. Sci-kit allows control of this penalty through the $\alpha$ parameter.

\paragraph{Elastic Net}
Elastic Net combines the penalties of Ridge Regression and LASSO to provide a tradeoff between the two functionalities. This tradeoff can be controlled through the l1\_ratio parameter.

\paragraph{Cross-validation}
Sci-kit also provides versions of these models which automatically apply cross-valdiation to determine the best values for their resepctive parameters.

\subsubsection{Support Vector Machines}
A support vector machine, at the simplest level, takes some input and assigns it to a binary class. At first this seems as linear as the previous models, but performing the kernel trick allows the SVM to deal with non-linear data. The choice of kernel can significantly affect the performance of the model. Two SVM implementations have been considered for this problem.

\paragraph{Support Vector Classifier}
Sci-kit provides more than one Support Vector Classifier but SVC has been chosen due to its support of non-linear kernels. SVC uses the one-against-one approach for multiclass problems.

\paragraph{Support Vector Regression}
SVR is scit-kit's implementation of Support Vector Regression. It also supports non-linear kernels.

\subsubsection{Artificial Neural Network}
This model was chosen because it can model very complex relationships, which many of the previous models cannot. Like all models, it takes the input vector and produces an output. It does this by passing the input through a series of hidden layers. These hidden layers are created and calibrated during the training process. The library pybrain had to be used for this because sci-kit provides no implementation of an Artificial Neural Network.
Pybrain provides a wide variety of options, too many to be considered in this one implementation. Here are some of the options pybrain provides: Back propagation trainer, choice of hidden layer, choice of number of hidden layers.

\section{Training and optimising}
Determining the best training methods and model parameters are tasks which can always be improved. Given the timescale of the project and that need to also integrate the model with the system, there are some model parameters and optimisation methods which have not been thoroughly investigated.

\subsection{K-fold cross validation}

\section{Results}

\section{Conclusion}

